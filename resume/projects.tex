\cvsection{Réalisations}

\begin{cventries}
  \cventry
    {} % Empty position
    {Home lab - mini data center} % Project
    {} % Empty location
    {} % Empty date
    {
      \begin{cvitems} % Description(s) bullet points
        \item {Infrastructure virtualisée pour implémenter des services variés de manière flexible}
        \item {Technologies actuellement utilisées : hyperviseur XCP-NG, Ubuntu Server configurées avec cloud-init, Kubernetes, Prometheus, Grafana,\\PfSense, Docker, DynDNS}
        \item {Services kubernetes internes : ArgoCD, Sealed-secrets, MetalLB, OpenEBS (stockage ZFS local),\\cert-manager, loki, prometheus \& promtail, nginx}
        \item {Technologies précédament utilisées : hyperviseur Proxmox, Reverse proxy NGINX, Windows server, PRTG, conteneurs linux (LXC)}
      \end{cvitems}
    }
  \cventry
      {} % Empty position
      {Projet Data Analysis Python Pandas \& MatPlotLib} % Project
      {} % Empty location
      {} % Empty date
      {
        \begin{cvitems} % Description(s) bullet points
          \item {Manipulation de données d'un dataset trouvé sur internet}
          \item {Résultat final : PDF contenant divers graphiques représentant les données du dataset\\}
        \end{cvitems}
      }
  \cventry
      {} % Empty position
      {Outils pour Kubernetes et Argocd} % Project
      {} % Empty location
      {} % Empty date
      {
        \begin{cvitems} % Description(s) bullet points
          \item {\href{https://github.com/guillheu/kubernetes-sealed-secrets-key-archiver}{Archivage des clés de chiffrement de Sealed-secret}}
          \item {\href{https://github.com/guillheu/argocd-kustomize-helm-cmp}{Plugin ArgoCD pour utiliser à la fois Kustomize et Helm}}
        \end{cvitems}
      }
  \cventry
    {} % Empty position
    {DeSign - Decentralized Signature} % Project
    {} % Empty location
    {} % Empty date
    {
      \begin{cvitems} % Description(s) bullet points
        \item {Application de signature simple (selon la norme européenne EIDAS) de documents sur blockchain Ethereum}
        \item {Projet de fin de parcours de la formation Développeur Blockchain chez Alyra (Fait avec le chef de projet Fawzi Benhalima)}
        \item {Application Java déployée directement chez le client}
        \item {\href{https://github.com/guillheu/DeSign}{https://github.com/guillheu/DeSign}}
      \end{cvitems}
    }
 
  \cventry
    {} % Empty position
    {Virtualisation Windows avec GPU \& NVME \& CPU passthrough sous linux} % Project
    {} % Empty location
    {} % Empty date
    {
      \begin{cvitems} % Description(s) bullet points
        \item {Machine virtuelle Windows avec des performances graphiques et disques, equivalentes à une installation "bare metal" pour une expérience de jeu optimale sous environnement Linux}
        \item {Découverte des groupes IOMMU ; compilation, patching (ACS) et installation d'un noyau linux ; configuration d'une machine virtuelle KVM-Qemu\\}
      \end{cvitems}
    }
  \cventry
    {} % Empty position
    {Controlleur MIDI sur Arduino Leonardo} % Project
    {} % Empty location
    {} % Empty date
    {
      \begin{cvitems} % Description(s) bullet points
        \item {Développement en C sur l'IDE Arduino}
        \item {Développement de scripts "Control Surface" en Python pour Ableton Live 10\\}
      \end{cvitems}
    }
  \cventry
    {} % Empty position
    {ChatBots : robots de salons de discussion} % Project
    {} % Empty location
    {} % Empty date
    {
      \begin{cvitems} % Description(s) bullet points
        \item {Programmes accédant à diverses plateformes de discussions via des API spécialisées, en Python et Java}
        \item {Délivrent des commandes du type "!ping", "!rules", "!numbers"}
        \item {Développés pour les plateformes Twitch et Discord\\}
      \end{cvitems}
    }
  \cventry
    {} % Empty position
    {HackerPI : station de "hacking" portable sur RaspberryPI Zero W} % Project
    {} % Empty location
    {} % Empty date
    {
      \begin{cvitems} % Description(s) bullet points
        \item {Découverte du "penetration testing" via le Re4son Kernel (Kali linux) pour raspberry PI}
        \item {\href{https://re4son-kernel.com/re4son-pi-kernel/}{https://re4son-kernel.com/re4son-pi-kernel/}}
        \item Brute force de codes PIN de téléphones Android. Contribution au repo github https://github.com/aagallag/hid\_gadget\_test\\pour la bonne gestion de l'entrée de chiffres via clavier virtuel\
      \end{cvitems}
    }
\end{cventries}